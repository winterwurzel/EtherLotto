%%%%%%%%%%%%%%%%%%%%%%%%%%%%%%%%%%%%%%%%%%%%%%%%%%%%%%%%%%%%%%%%%%%%%%%%%%%%%
\chapter{Introduction}\label{chap:intro}
%%%%%%%%%%%%%%%%%%%%%%%%%%%%%%%%%%%%%%%%%%%%%%%%%%%%%%%%%%%%%%%%%%%%%%%%%%%%%
\chapterstart

Ethereum is a cryptocurrency similar to Bitcoin. Like Bitcoin, it has a blockchain to guarantee that transactions from one user to another are secure and safe.
The blockchain holds all transactions in a linear, time-stamped series of bundled transactions known as blocks and the calculation of each block is depending on the one before. (mining)
Essentially a blockchain is like a public ledger for recording transactions which is freely shared, continually updated and most important: under no central control.

The Ethereum cryptocurrency coin is called Ether. To make transactions on teh Ethereum Blockchain a certain fee is required to be paid. This fee is called Gas and can be set to a maximum amount before confirming and sending a transaction by the User.

Ethereum has its own implementation of a blockchain which expands the capabilities with Smart Contracts.

A Smart Contract is basically code written into the blockchain. Therefore it is public and everyone can read the source code and know the specifics of the contract, while the individuals involved are anonymous.

Some time after the Smart Contract is formed between the two parties a triggering event occurs. This can be pretty much anything, for example a expiration date. Now the contract executes itself according to the coded terms and enforces automatically all obligations. This automatic enforcing is the key difference to traditional contracts.


\chapterend

