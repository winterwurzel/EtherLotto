%%%%%%%%%%%%%%%%%%%%%%%%%%%%%%%%%%%%%%%%%%%%%%%%%%%%%%%%%%%%%%%%%%%%%%%%%%%%%
\chapter{Related Work}\label{chap:relatedwork}
%%%%%%%%%%%%%%%%%%%%%%%%%%%%%%%%%%%%%%%%%%%%%%%%%%%%%%%%%%%%%%%%%%%%%%%%%%%%%
\chapterstart

Ethereum is a cryptocurrency similar to Bitcoin. Like Bitcoin, it has a blockchain to guarantee that transactions from one user to another are secure and safe. The blockchain holds all transactions in a linear, time-stamped series of bundled transactions known as blocks and the calculation of each block is depending on the one before. Essentially, a blockchain is like a public ledger for recording transactions. This ledger is freely shared, continually updated and most important, under no central control.

The Ethereum cryptocurrency coin is called Ether. To make transactions on the Ethereum Blockchain a certain fee is required to be paid. This fee is called Gas and can be set to a maximum amount before confirming and sending a transaction by the User.
Ethereum has its own implementation of a blockchain which expands the capabilities with Smart Contracts.

Smart Contracts are computerized transaction protocols which autonomously execute the terms of a contract. \citep[cf.]{Giancaspro:2017} states that even though smart contracts have the potential to increase commercial efficiency, reduce transaction and legal costs, and facilitate transparent and anonymous transacting, there are still questions regarding the legal enforceability of smart contracts. At this point, it is uncertain whether smart contracts can easily be adapted to fit into current legal frameworks, which are used to regulate traditional contracts.

Smart contracts also pose a certain security thread based on the logic they are coded with. According to \citep[cf.]{DAKMS:2015}, there are three major logic problems which often occur while writing a smart contract. 
Contracts do not refund. Contracts hold all the money sent to them and some can only progress further if the user sends a certain amount of Ether. If a user sends less than required by the contract, a refund might not happen.
Lack of cryptography to achieve fairness. Some contracts perform calculations based on a user’s input to decide the outcome. These inputs have to be stored encrypted to prevent malicious users from submitting inputs biased in their favor. 
Incentive misalignment. Some contracts do not incentivize users to follow intended behavior. Consider a gambling game that uses a commit-reveal scheme in which participants first submit their encrypted move along with a deposit before later revealing it. After the first move is revealed, the second user may already realize his move will lose. Since his deposit lost, he may not be willing to spend gas to reveal his choice.
    
In conclusion we can see clearly, that there are legal as well as logic problems to be solved before smart contracts can be used in large scale businesses. The legal status of smart contracts is still uncertain since new laws need to be put in place or existing need to be adapted. Whereas the logic problems can be mitigated by applying different rulesets during development and with the growing maturity of the smart contract code framework in Ethereum.

Sources: \newline
\citetitle{Giancaspro:2017}\newline
\citetitle{DAKMS:2015}\newline
\citetitle{Mohan:2017}

\chapterend

